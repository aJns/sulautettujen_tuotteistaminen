\section{Johdanto}
Langattomat anturiverkot ovat IOT:n kehityksen myötä yleistyneet monilla
aloilla. Anturien ja niiden keräämän datan lisääntyminen on luonut haasteita
verkkojen tehokkaalle toiminnalle. Tämän lisäksi niiden liikkuvuus erilaisten
alustojen ja robottien mukana on liisääntymässä, joka tuo omat haasteensa
niiden kehitykselle.

Tässä työssä käsitellään vahvistusoppimisen sovelluksia anturiverkkojen
verkkoliikenteen optimoinnille, ja kuinka se vaikuttaa tuotekehitykseen. Työ
perustuu pitkälti lähteiden~\cite{Arya2015} ja~\cite{Yu2006} tuloksiin.
