\section{Johdanto}
Langattomat anturiverkot ovat juontavat juurensa Yhdysvaltojen asevoimien
tutkimusorganisaatio DARPA:n projekteihin jotka tutkivat hajautettujen
anturiverkkojen sovelluksia. Modernit anturiverkot ovat kuitenkin tulleet
pitkällä noista ajoista. Nykyään anturiverkot ovat langattomia, ne keräävät
entistä enemmän dataa ja staattisten mikroantureiden sijaan anturit ovat osa
esimerkiksi liikkuvia droneja.

Anturiverkot koostuvat suurimmaksi osaksi tietenkin antureista, mutta niissä on
yleensä vähintään tukiasema minne data lopulta kerätään. Tämän lisäksi
verkoissa voi olla reititykseen tarkoitettuja laitteita.  Tämän selvityksen
lähteissä käytetään usein termiä ``node'' tai anturiverkon solmu kuvaaman
verkon jäseniä.  Anturiverkon solmu tai pelkkä solmu on varsinkin reitityksestä
puhuessa anturia parempi termi koska reititykseen voi osallistua muitakin
laitteita kuin antureita. Reitityksessä ei myöskään ole yleensä väliä onko
kyseessä anturi vai joku muu palikka.

Anturiverkot toimivat yleensä ulkona ja laajalla alueella jossa niiden
toimintaa haittaa lika, sää, maantieteelliset esteet ja pahimmillaan jopa
vihamielinen toiminta. Antureita ei ole järkevä ladata manuaalisesti, joten
virrankulutus tulee minimoida. Haasteena on siis reitittää antureiden keräämä
data tehokkaasti ja luotettavasti, vaikka verkon solmut liikkuisivat tai niitä
tuhoutuisi.

Vahvistusoppiminen on yksi työkalu näiden ongelmien ratkomiseen.
Vahvistusoppimisen suurin etu reititysprotokollassa on että se on erittäin
mukautuva ja se tarvitsee minimaalisesti tietoa verkosta etukäteen.

Tässä työssä käsitellään vahvistusoppimisen sovelluksia anturiverkkojen
verkkoliikenteen optimoinnille, ja kuinka se vaikuttaa tuotekehitykseen. Työ
perustuu pitkälti lähteiden~\cite{Arya2015} ja~\cite{Yu2006} tuloksiin.
