\begin{abstract}

  Anturiverkot ovat kasvava ilmiö, joilla on useita sovelluskohteita. Edullisten
  ja pienten verkotettujen antureiden seurauksena on entistä helpompaa kerätä
  paljon dataa laajoiltakin alueilta. Modernit anturiverkot koostuvat
  staattisten mikroantureiden lisäksi myös liikkuvista droneista.  Antureiden
  keräämän datan määrä ja niiden vaikeat toimintaolosuhteet luovat haasteita
  verkkojen tiedonsiirrolle. Näiden haasteiden ratkaisuun tarvitaan älykkäitä
  ratkaisuja, joista yksi on vahvistusoppimisen hyödyntäminen reitityksessä.
  Vahvistusoppiminen on koneoppimisen alalaji joka pyrkii etsimään parhaimman
  toimintatavan toimintaympäristössään. Vahvistusoppimista hyödyntävät
  anturiverkot pystyvät sopeutumaan muuttuviin ja vaikeisiin olosuhteisiin ja
  reitittämään dataa tehokkaasti ja luotettavasti.


  \subsubsection*{Avainsanat}
  Internet of Things, langattomat verkot, langattomat
  anturiverkot, koneoppiminen, vahvistusoppiminen
\end{abstract}
