\section{Johtopäätökset}
\label{sec:conclusion}

Tämän kirjallisuusselvityksen pohjalta voidaan todeta että vahvistusoppimisella
on potentiaalia anturiverkkojen reitityksessä. Anturiverkkojen
toimintaolosuhteet vaativat reititysalgoritmeiltä mukautuvuutta ja tehokkuutta.
Näihin vaatimuksiin voidaan vastata vahvistusoppimisen menetelmillä.
Vahvistusoppimisen menetelmät sopivat useimpiin tilanteisiin. Vahvitusoppimista
hyödyntävät reititysalgoritmit myös rasittavat verkkoa vähemmän sillä
yksittäisten solmujen ei tarvitse olla tietoisia kaikista verkon jäsenistä.
Useissa tapauksissa riittää että ne tuntevat vain välittömät naapurinsa.
