\section{Kirjallisuusselvitys}

\subsection{Artikkeleiden keskeiset ongelmat}
\TODO{Selitä tarkasti mitä ongelmaa artikkelit ryhtyvät ratkaisemaan}

Anturiverkkojen toimintaolosuhteista johtuen verkon yhteydet ovat muuttuvia ja
ennalta-arvaamattomia. Usein verkon yksittäisillä antureilla ei ole tietoa
verkon koostumuksesta sen välittömien naapureiden lisäksi. Siksi
reititysprotokollien täytyy pystyä sopeutumaan muuttuviin tilanteisiin
mahdollisimman vähällä informaatiolla itse verkosta.

Artikkelissa~\cite{Yu2006} keskeistä on anturien keräämän datan fuusio.
Datafuusio tarkoittaa toisiinsa liittyvän datan liittämistä yhteen. Artikkeli
käyttää esimerkkinä UAV:sta koostuvaa anturiverkkoa, joissa jokaisella
yksittäisellä UAV:lla on matala varmuus kokonaistilanteesta. Datafuusio on
tärkeää jotta verkolla on vahva käsitys tilanteesta jonka avulla se voi
suunnitella toimintaansa.

Anturiverkon osien, eli "nodejen" kuolemat ovat ongelma jonka verkon täytyy
pystyä kestämään ja käsittelemään. Kuolema tarkoittaa tässä kontekstissa mitä
tahansa tilannetta jossa node on toimintakyvytön tai sen toimintakyky on niin
huono että siitä ei ole hyötyä verkossa. Artikkelissa~\cite{Arya2015} esitelty
algoritmi pyrkii etsimään nodejen kuolemista syntyviä "reikiä" ja reitittämään
data optimaalisesti ne huomioon ottaen.

\subsection{Artikkeleiden tarjoamat ratkaisut}
\TODO{Selitä tarkasti miten em ongelma ratkaistaan}


