\section{Kirjallisuusselvitys}

\subsection{Tarve tiedonsiirron optimoinnille}
Anturiverkoissa tarvitaan tiedonsiirtoalgoritmeja niiden suurista datamääristä
johtuen. Usein anturiverkoista on myös tärkeää saada reaaliaikaista dataa, joka
tarkoittaa tarvetta minimaaliselle latenssille. Ilman tehokasta datan
reititystä verkko tukkiutuisi suuren tietoliikenteen seurauksena, ja tärkeä
anturidata saattaisi kulkea perille turhan hidasta reittiä. 

\subsection{Datan käsittely keskitetysti vs.\ hajautetusti}
Kerätty anturidata voidaan lähettää keskeiselle tiedonkäsittely yksikölle.
Hyvänä puolena tässä on se että antureiden toteutus voi olla minimaalinen koska
tiedonkäsittelykapasiteettia ei tarvitse olla, mutta toisaalta raa'an datan
puskeminen verkon läpi rasittaa verkkoa turhaan.

Hajautetussa käsittelyssä jokainen anturi pystyy käsittelämään keräämäänsä
dataa jonkun verran. Vaihtoehtoisesti verkkoon lisätään tiedonkäsittely
yksiköitä, jotka ovat erikoistuneet tehtävään, mutta ovat kuitenkin
yksinkertaisempia kuin edellä mainittu keskitetty vaihtoehto. Hajautettu
käsittely monimutkaistaa antureiden toteutusta, mutta vähentää verkon rasitusta
ja myös tekee verkosta kestävämmän koska yksittäistä heikkoa kohtaa ei ole.

\subsection{Olemassa olevia tiedonsiirtoalgoritmeja}

\subsection{Vahvistusoppiminen tiedonsiirtoverkon optimoinnissa}

\subsection{Vaikutus tuotekehitykseen}
