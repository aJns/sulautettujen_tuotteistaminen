\section{Käsitteiden määrittely}

\begin{description}

  \item [Internet of things, IoT] Suomennettuna ``esineiden internet'', usein
    käytetään lyhennettä IoT. IoT on laaja konsepti, ja tieteen ja teollisuuden
    kirjallisuudessa on jonkun verran epäselvyyttä siitä mitä termi pitää
    sisällään. IoT:ssa on kumminkin keskeistä erilaisten esineiden (things)
    verkottaminen toisten esineiden ja laajemman internetin kanssa. Keskeistä
    on myös että nämä esineet toimivat yhteistyössä keräämällä, käsittelemällä
    ja toimimalla datan perusteella.~\parencite{Atzori2010a, Gubbi2013}

  \item [Anturiverkot] koostuvat toisiinsa linkitetyistä antureista. Nykyään
    puhutaan käytännössä langattomista verkoista (Wireless Sensor Network,
    WSN).  Anturit voivat pysyä paikallaan tai olla liikkuvia, esimerkiksi
    robotteja tai droneja. Antureiden keräämä data voidaan käsitellä joko
    keskitetysti, tai sitten hajautetusti.~\parencite{Chong2003, Tubaishat2003}

  \item [Koneoppiminen] (machine learning) Tarkoittaa erilaisten mallien
    luomista datan perusteella. Käytetään tilanteissa joissa ihminen ei kykene
    luomaan matemaattista mallia, johtuen yleensä datan määrästä ja
    monimuotoisuudesta.  Termi käsittää useampia alatermejä jotka voivat olla
    hyvinkin erilaisia.~\parencite{Grosan2011}

  \item [Vahvistusoppiminen] (reinforcement learning) on koneoppimisen alalaji,
    josta käytetään myös nimitystä ``robottioppiminen'', koska se on sen
    yleisin sovelluskohde.  Vahvistusoppimisessa koneoppimismallia muokataan
    ympäristön ja tilanteen mukaan. Vahvistusoppiminen perustuu oppijan
    (robotti, tekoäly tms) palkitsemiseen tai rankaisemiseen riippuen siitä
    kuinka hyvin se suorittaa sille annetun tehtävän (reitin etsiminen, pelin
    pelaaminen tms). Palkkion ja rankaisun tulisi vaikuttaa oppijan toimintaan
    niin että sen suoriutuminen tehtävästä paranee.~\parencite{Kaelbling1996}

\end{description}
