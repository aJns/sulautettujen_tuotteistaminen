\section{Käsitteiden määrittely}

\begin{description}

  \item [Internet of Things, IoT]
    Suomennettuna ``esineiden internet'', yleensä
    käytetään lyhennettä IoT. Tarkoittaa toisiinsa linkitettyjä ``esineitä'', eli
    laitteita joiden kummassakaan päässä ei ole ihmistä, vrt.\ normaalin internetin
    toimintaan, jossa yleensä ainakin yhdessä päässä kaikkia interaktiota on
    ihminen.

  \item [Anturiverkot] koostuvat toisiinsa linkitetyistä
    antureista. Nykyään puhutaan käytännössä langattomista verkoista.
    Anturit voivat pysyä paikallaan tai olla liikkuvia, esimerkiksi robotteja
    tai droneja. Antureiden keräämä data voidaan käsitellä joko keskitetysti,
    tai sitten hajautetusti.~\cite{Chong2003,Tubaishat2003}

  \item [Koneoppiminen]
    Tarkoittaa erilaisten mallien luomista datan perusteella. Käytetään
    tilanteissa joissa ihminen ei kykene luomaan matemaattista mallia, johtuen
    yleensä datan määrästä ja monimuotoisuudesta. Termi käsittää useampia
    alatermejä jotka voivat olla hyvinkin erilaisia.

  \item [Vahvistusoppiminen] on koneoppimisen alalaji, josta käytetään myös
    nimitystä ``robottioppiminen'', koska se on sen yleisin sovelluskohde.
    Vahvistusoppimisessa koneoppimismallia muokataan ympäristön ja tilanteen
    mukaan. Vahvistusoppiminen perustuu oppijan (robotti, tekoäly tms)
    palkitsemiseen tai rankaisemiseen riippuen siitä kuinka hyvin se suorittaa
    sille annetun tehtävän (reitin etsiminen, pelin pelaaminen tms). Palkkion
    ja rankaisun tulisi vaikuttaa oppijan toimintaan niin että sen
    suoriutuminen tehtävästä paranee.~\cite{Kaelbling1996}

\end{description}
